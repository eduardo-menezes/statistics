\documentclass[12pt,a4paper]{book}

\newcommand{\overbar}[1]{\mkern 1.5mu\overline{\mkern-1.5mu#1\mkern-1.5mu}\mkern 1.5mu}


\usepackage[ left = 2cm, right = 3cm, top = 3cm, bottom = 3cm]{geometry}
\usepackage[utf8]{inputenc}
\usepackage[T1]{fontenc}
\usepackage[brazil]{babel}
\usepackage{amsmath}
\usepackage{amsfonts}
\usepackage{amssymb}
\usepackage{graphicx}
\usepackage{float}
\usepackage{natbib}
\usepackage{bm}
\usepackage[short]{optidef}
\usepackage{cancel}
\usepackage{comment}
\usepackage{siunitx}
\usepackage{indentfirst}
\usepackage[colorlinks = true,
linkcolor = blue,
urlcolor  = blue,
citecolor = blue,
anchorcolor = blue]{hyperref}
\usepackage{fancyhdr}
\pagestyle{fancy}
\fancyhf{}


\begin{document}	
	
	
	\fancyhead[OR,EL]{\thepage}
	%\lhead{\rightmark}
	\setcounter{page}{1}
	

	\bibpunct{(}{)}{;}{a}{,}{,}
	
	\chapter{CORRELAÇÃO}
	
		\textbf{Correlação} indica a força de associação entre duas variáveis. A correlação não implica causa sendo que modo de representá-la é utilizando gráficos de dispersão.
	
		\section{Medidas de Correlação}
				Dadas $n$ observações e sendo $ x $ e $ y $ variáveis, calcula-se
				\subsection{ médias amostrais $\bar{x}$ e $\bar{y}$}
				
				\begin{eqnarray}
					\bar{x} = \frac{\sum x_i}{n} \\
					\bar{y} = \frac{\sum y_i}{n}
				\end{eqnarray}
			
				\subsection{Variância Amostral $S_{xx}$ e $S_{yy}$}
				
				\begin{eqnarray}
					S_{xx} = \frac{1}{n}\sum(x_i - \bar{x})^2 \\
					S_{yy} = \frac{1}{n}\sum(y_i - \bar{y})^2
				\end{eqnarray}
			
				\subsection{Covariância Amostral $S_{xy}$}
				
				\begin{eqnarray}
					S_{xy} = \frac{1}{n}\sum(x_i - \bar{x})(y_i - \bar{y})
				\end{eqnarray}
				
				\subsection{Correlação de Pearson $r_{xy}$}
				\begin{eqnarray}
					r_{xy} = \frac{\sum x_i y_i - n\bar{x}\bar{y}}{\sqrt{(\sum x_{i}^2 - n\bar{x}^2)}\sqrt{(\sum y_{i}^2 - n\bar{y}^2)}}
				\end{eqnarray}
				
	
	
	\newpage 
	\centering
	\bibliographystyle{plainnat}

	\bibliography{referencias}


\end{document}